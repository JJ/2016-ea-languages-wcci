\documentclass[sigconf]{acmart}\usepackage[]{graphicx}\usepackage[]{color}
%% maxwidth is the original width if it is less than linewidth
%% otherwise use linewidth (to make sure the graphics do not exceed the margin)
\makeatletter
\def\maxwidth{ %
  \ifdim\Gin@nat@width>\linewidth
    \linewidth
  \else
    \Gin@nat@width
  \fi
}
\makeatother

\definecolor{fgcolor}{rgb}{0.345, 0.345, 0.345}
\newcommand{\hlnum}[1]{\textcolor[rgb]{0.686,0.059,0.569}{#1}}%
\newcommand{\hlstr}[1]{\textcolor[rgb]{0.192,0.494,0.8}{#1}}%
\newcommand{\hlcom}[1]{\textcolor[rgb]{0.678,0.584,0.686}{\textit{#1}}}%
\newcommand{\hlopt}[1]{\textcolor[rgb]{0,0,0}{#1}}%
\newcommand{\hlstd}[1]{\textcolor[rgb]{0.345,0.345,0.345}{#1}}%
\newcommand{\hlkwa}[1]{\textcolor[rgb]{0.161,0.373,0.58}{\textbf{#1}}}%
\newcommand{\hlkwb}[1]{\textcolor[rgb]{0.69,0.353,0.396}{#1}}%
\newcommand{\hlkwc}[1]{\textcolor[rgb]{0.333,0.667,0.333}{#1}}%
\newcommand{\hlkwd}[1]{\textcolor[rgb]{0.737,0.353,0.396}{\textbf{#1}}}%
\let\hlipl\hlkwb

\usepackage{framed}
\makeatletter
\newenvironment{kframe}{%
 \def\at@end@of@kframe{}%
 \ifinner\ifhmode%
  \def\at@end@of@kframe{\end{minipage}}%
  \begin{minipage}{\columnwidth}%
 \fi\fi%
 \def\FrameCommand##1{\hskip\@totalleftmargin \hskip-\fboxsep
 \colorbox{shadecolor}{##1}\hskip-\fboxsep
     % There is no \\@totalrightmargin, so:
     \hskip-\linewidth \hskip-\@totalleftmargin \hskip\columnwidth}%
 \MakeFramed {\advance\hsize-\width
   \@totalleftmargin\z@ \linewidth\hsize
   \@setminipage}}%
 {\par\unskip\endMakeFramed%
 \at@end@of@kframe}
\makeatother

\definecolor{shadecolor}{rgb}{.97, .97, .97}
\definecolor{messagecolor}{rgb}{0, 0, 0}
\definecolor{warningcolor}{rgb}{1, 0, 1}
\definecolor{errorcolor}{rgb}{1, 0, 0}
\newenvironment{knitrout}{}{} % an empty environment to be redefined in TeX

\usepackage{alltt}
%%% Local Variables:
%%% ispell-local-dictionary: "english"
%%% End:

\usepackage[utf8]{inputenc}
\usepackage{booktabs} % For formal tables
\usepackage{graphicx}
\usepackage{rotating}
\usepackage{listings}

\definecolor{Gray}{gray}{0.6}

\copyrightyear{2018}
\acmYear{2018}
\setcopyright{acmlicensed}
\acmConference[GECCO '18 Companion]{Genetic and Evolutionary Computation Conference Companion}{July 15--19, 2018}{Kyoto, Japan}
\acmBooktitle{GECCO '18 Companion: Genetic and Evolutionary Computation Conference Companion, July 15--19, 2018, Kyoto, Japan}
\acmPrice{15.00}
\acmDOI{10.1145/3205651.3208273}
\acmISBN{978-1-4503-5764-7/18/07}


\title{Performance improvements of evolutionary algorithms in Perl 6} 

\author{Juan-Julián Merelo-Guervós}
\orcid{1234-5678-9012}
\affiliation{%
  \institution{Universidad de Granada}
  \streetaddress{Daniel Saucedo Aranda, s/n}
  \city{Granada}
  \country{Spain}
}
\email{jmerelo@ugr.es}

\author{José-Mario García-Valdez}
\affiliation{%
  \institution{Instituto Tecnológico de Tijuana}
  \streetaddress{Calzada Tecnológico, s/n}
  \city{Tijuana}
  \country{Mexico}
}
\email{mario@tectijuana.edu.mx}
% The default list of authors is too long for headers.
%\renewcommand{\shortauthors}{J. J. Merelo et al.}

\begin{abstract}
  
  Perl 6 is a recently released language that belongs to the Perl family but was
  actually designed from scratch, not as a refactoring of the Perl 5 codebase. Through its two-year-old
  (released) history, it has increased performance by several orders
  of magnitude, arriving recently to the point where it can be safely
  used in production. In this paper, we are going to compare the
  historical and current performance of Perl 6 in a single problem,
  OneMax, to those of other interpreted languages; besides, we will
  also use implicit concurrency and see what kind of performance and
  scaling can we expect from it.  
  
\end{abstract}

\begin{CCSXML}
<ccs2012>
<concept>
<concept_id>10003752.10003809.10003716.10011136.10011797.10011799</concept_id>
<concept_desc>Theory of computation~Evolutionary algorithms</concept_desc>
<concept_significance>500</concept_significance>
</concept>

<concept>
<concept_id>10010147.10010919.10010172</concept_id>
<concept_desc>Computing methodologies~Distributed algorithms</concept_desc>
<concept_significance>300</concept_significance>
</concept>
</ccs2012>
\end{CCSXML}

\ccsdesc[500]{Theory of computation~Evolutionary algorithms}

\ccsdesc[300]{Computing methodologies~Distributed algorithms}

\keywords{Benchmarking, computer languages, concurrency, evolutionary
  algorithms, Perl, Perl 6}

\maketitle


\section{Introduction}

Performance has always been a concern in scientific
computing. Generally, you will want to use the fastest language
available to be able to run your experiments in as little time as
possible. However, while implementation matters
\cite{DBLP:conf/iwann/MereloRACML11}, ease of programming, available
libraries and supporting community are sometimes more significant concerns,
since in scientific computing the target is to optimize
time-to-publish the paper, not only time from pressing {\em Enter} to
obtaining the results, and that includes time to get toe program done
itself, as well as process results.

In that sense, interpreted languages such as Python, Perl or JavaScript
\cite{fortin2012deap,DBLP:conf/ijcci/FarisAMCM16,DBLP:conf/gecco/GuervosVGES14,perl-ea,hidaka2017development,rivas2014object,ae09}
offer fast prototyping, if not the fastest implementation, which
usually belongs to compiled languages such as Haskell or Java
\cite{DBLP:conf/evoW/MereloCBRGFRV16}. However, as proved in the cited
paper, that is not always the case and new languages deserve a chance
to be tested, mainly if they offer functionalities that might make the
implementation of evolutionary algorithms faster or more straightforward.

Besides, the performance of a language is not a static thing; while some
languages are happy enough with the level they achieve and focus on
other functionalities, newer languages focus on performance in every
new release, offering improvements of several orders of
magnitude. This has been the case of Perl 6
\cite{Tang:2007:PRI:1190216.1190218}, a new, concurrent, dynamic and
multi-paradigm language that has been in development since 2000 and
released in December 2015. Since then, it has had a release cycle of
one, or sometimes more, releases every month, with a stable release
every four months. While initial tests discouraged us from including
its figures in the paper where we benchmarked many languages for
evolutionary algorithms \cite{DBLP:conf/evoW/GuervosBCRGRVHR17}, the
increase in performance has been continuous, as well as the
implementation of implicit parallelism features. 

This paper is specially focused on benchmarking this language for
evolutionary algorithms, with the intention of proposing it as
production-ready for scientific computing or evolutionary computation
experiments. 

The rest of the paper is organized as follows. We will briefly present
the state of the art in benchmarking evolutionary algorithms in the
next section, followed by the set of experiments used to test
the performance in Section \ref{sec:exp}. Results and charts will be
presented in Section \ref{sec:res}, and we will close the paper by
stating our conclusions. 

\section{State of the art}
\label{sec:soa}

As a matter of fact, there is very little scientific literature on
Perl 6, much less applied to scientific computing. The paper by Audrey
Tang \cite{Tang:2007:PRI:1190216.1190218}, one of the early
programmers of a Perl 6 compiler in Haskell called Pugs, is one of the
few we can find. In fact, the paper where she describes the design of
the language has had some influence in language design, including the
design of Typed Scheme, a functional language \cite{tobin2008design}. 

Its sister language, Perl, has been used in
Evolutionary Algorithms for a long time, with an early tool but
used for minimizing the performance of a network
\cite{bucur2016optimizing}. Since the publication of the {\tt
  Algorithm::Evolutionary} library circa 2002
\cite{ecperl,perl-ea} it has been applied to many different problems,
including solving the MasterMind puzzle \cite{DBLP:journals/evi/Maestro-MontojoSG14}. 
In fact, its speed processing evolutionary
algorithms has made it a great tool for evolving regular expressions
via the DiscoverRegex and GenRegex tools \cite{ruano2018using}, and
even optimizing the yield of analog integrated circuits \cite{guerra2015ocba}. 

Perl 5 was a convenient and multi-paradigm, if not particularly
groundbreaking language. Conceptually, you could program an
evolutionary algorithm in pretty much the same way you would do it in
C or C++, which were at the time much faster languages. The fact that
it was used proves that languages for implementing evolutionary
algorithms are not chosen purely by their raw speed. 

However, speed has to be adequate and not vary in orders of magnitude
with respect to other, well-established, language. Even if slower, the
trade-off might be interesting if a new language offers new ways of
% might be worthy?
implementing evolutionary algorithms that give you some insight on the
inner workings of evolutionary optimization. This why in this paper we
will set to measure the speed of Perl 6 and its evolution, in order to
prove that it has come the time to consider it as a language for
evolutionary optimization given the functional and concurrent
facilities it now offers. 

% Maybe add a comment about how much different are Perl 5 and Perl 6?
% They are compatible? Can we use parts of Algorithm::Evolutionary?

\section{Experimental setup}
\label{sec:exp}

In this experiment we have used the same operators and data already
published in \cite{DBLP:conf/evoW/MereloCBRGFRV16} that is, crossover,
mutation and one-max or count-ones. We have added the Royal Road
function \cite{mitchell1992royal}, mainly with the objective of
comparing Perl and Perl 6 and its parallel facilities. 

The functions are well known, and the main objective of these tests
was, besides comparing performance side by side, see how this
performance scales to big, and a bit unrealistic, chromosome
sizes. The way the handling of data structures by particular languages
is done makes that, sometimes, the speed of dealing with bigger sizes
is faster than with smaller sizes; as a matter of fact, in the above
mentioned paper Java achieved its best performance for the biggest chromosome
size. 
% Big chromosome -> large chromosome?
We tested several data structures in Perl 6 and finally chose a vector
(or array) of booleans as the fastest one. In fact the speed of the
benchmark is divided in two parts: speed for randomly generating the
vector and speed of actually counting the number of ones. In this
case, generating a vector of {\tt Bool}s was considerably faster than
doing the same with integers, although summing them was almost 4 times
as slow. That is why we also test a vector of integers in the
experiments we show below. These two operations take two lines in Perl
6, as follows.

\begin{lstlisting}[language=Perl]
my $ones = Bool.roll xx $len ;
my $maxones = $ones.sum;	
\end{lstlisting}

These two lines show the advantage of this kind of language; the same
operation needs several lines and two loops in most other,
non-functional, languages. The first one creates an array of random
boolean values, generated with {\tt Bool.roll}; {\tt xx} {\em
  multiplies} by the length to yield an array of the desired
length. And the second line just uses the {\tt sum} method, which is
an standard method for arrays and can also be applied to arrays of
booleans. In Perl 6, there are many possible ways of achieving the
same, but in fact after several measurements we found this was the
fastest, even if initially it was much slower than for other
languages. Also, as it can be seen, Perl 6 uses {\em sigils} for
variables, this {\tt \$} been applied to most kinds of containers. % has been?
 {\tt
  Bool} is a standard type, and {\tt my} is a {\em scope} declaration
which can optionally include a type or class declaration. A fuller
introduction to the language is outside the scope of this paper, but
the interested reader can check the documentation at
\url{https://docs.perl6.org} for a tutorial or a more thorough
explanation of all its features and capabilities. 

The benchmark consists in 100,000 repetitions of the operation for
sizes that are increased by 2 starting from 16 to, when possible, 32768. All
experiments took place in a desktop computer with 8 cores running
Ubuntu 14.04.5.

All programs are open source, and included in the same GitHub repository that
holds this paper in \url{https://github.com/JJ/perl6eo}. Data from the experiments is also
freely available in the same place. 

\section{Results and analysis}
\label{sec:res}
%
\begin{figure}[h!tb]
  \centering
\begin{knitrout}
\definecolor{shadecolor}{rgb}{0.969, 0.969, 0.969}\color{fgcolor}
\includegraphics[width=\maxwidth]{figure/results-perl6-1} 

\end{knitrout}
\caption{Plot of time needed to perform 100K OneMax evaluations in
 several versions of Perl 6, from 2016 to the current in 2018.
 Strings have 
lengths increasing by a factor of two from 16 to $2^{15}$. Please note
that $x$ and $y$ both have a logarithmic scale.}
\label{fig:perl6:mo}
\end{figure}
%

The first experiment just measured the speed of the {\sf max-ones}
function across releases of Perl 6; Perl 6 has a monthly release
schedule, with version number corresponding to year and month. The
result of this operation is shown in Figure \ref{fig:perl6:mo}, and it
clearly shows the increase in speed across time, that amounts to
almost one order of magnitude from the first version, with a
performance that prompted us to exclude it from our initial study, to
the current, which is much better. 
%
\begin{figure*}[h!tb]
  \centering
\begin{knitrout}
\definecolor{shadecolor}{rgb}{0.969, 0.969, 0.969}\color{fgcolor}
\includegraphics[width=\maxwidth]{figure/results-mo-1} 

\end{knitrout}
\caption{Plot of time needed to perform 100K OneMax function evaluations in strings with
lengths increasing by a factor of two from 16 to $2^{15}$. Please note
that axes $x$ and $y$ both have a logarithmic scale.}
\label{fig:time:mo}
\end{figure*}
%

Despite the improvement, it needs to be compared to the rest of the
languages we tested in the previous paper. We have excluded the
fastest, mainly compiled, languages, to leave mainly scripting, and
some compiled, languages. This comparison is shown in Figure
\ref{fig:time:mo}. This chart, besides all the measures already
published in the previous paper, includes three versions of the
one-max in Perl 6. One is the same as above, which uses a boolean
representation for the chromosome bits; the second uses an integer
representation for the bits and is listed as {\tt IntVector}. This
version needed a bit of hacking which included using a Boolean bit
generation and then transforming it to an integer number; however,
even that step made it a bit slower than the Boolean version.

%
\begin{figure*}[h!tb]
  \centering
\begin{knitrout}
\definecolor{shadecolor}{rgb}{0.969, 0.969, 0.969}\color{fgcolor}
\includegraphics[width=\maxwidth]{figure/results-bf-1} 

\end{knitrout}
\caption{Plot of time needed to perform mutation on  100K chromosomes
 with increasing lengths from 16 to $2^{15}$. Please note
that $x$ and $y$ both have a logarithmic scale.}
\label{fig:time:bf}
\end{figure*}
%
The third version, listed as {\tt perl6-BitVector-hyper}, shows one of
the unique characteristics of Perl 6: implicit parallelism. The {\tt
  hyper} and {\tt race} methods, applied to vectors, divide the job
into different threads, 4 by default, evaluating different parts of
the vector in parallel, without affecting in any way the rest of the
operation. In the case above, just changing the line to 
\begin{lstlisting}[language=Perl]
my $maxones = $ones.race.sum;	
\end{lstlisting}
made the sum to be executed in parallel, improving the performance by
the number of threads it is using by default. We used {\tt race}
instead of {\tt hyper} since the latter forces in-order execution; in
our case, the order of the sums is not important and keeping order
makes it a bit slower.

The chart shows that, in fact, Perl 6 for this operation is faster,
for big sizes, than C++, and overall faster than the Lua language or
even Python for a particular representation. For some sizes, it can
also be faster than Common Lisp. 
%
\begin{figure*}[h!tb]
  \centering
\begin{knitrout}
\definecolor{shadecolor}{rgb}{0.969, 0.969, 0.969}\color{fgcolor}
\includegraphics[width=\maxwidth]{figure/results-xo-1} 

\end{knitrout}
\caption{Plot of time needed to perform crossover on 100K chromosomes
 with increasing lengths from 16 to $2^{15}$. Please note
that $x$ and $y$ both have a logarithmic scale.}
\label{fig:time:xo}
\end{figure*}
%

In principle, by being faster than more traditional languages, we can
prove here that Perl 6 can be not only convenient in terms of
programming ease (just two lines where other languages need many more
lines), but also faster. Let us, however, have a look at the rest of
the genetic operations. 

The very traditional bitflip mutation comparison chart is shown in
Figure \ref{fig:time:bf}. The lines used for doing this operation are
shown below.
\begin{lstlisting}[language=perl]
my $position = $range.pick;
@ones[$position] = !@ones[$position];
\end{lstlisting}
In this case we are using {\tt pick} for choosing a random value in a
range, which is the chromosome size, and flipping the bit in that
random position. Could be done also in a single line, avoiding the
{\tt \$position} variable; besides, we use the {\tt \@} sigil to
clearly indicate we are dealing with a vector. We avoided it in the
listing above since it made the operation slightly slower. 

%
\begin{figure*}[h!tb]
  \centering
\begin{knitrout}
\definecolor{shadecolor}{rgb}{0.969, 0.969, 0.969}\color{fgcolor}
\includegraphics[width=\maxwidth]{figure/results-rr-1} 

\end{knitrout}
\caption{Plot of time needed to perform 100K royal road functions on chromosomes
 with increasing lengths from 16 to $2^{15}$. Please note
that $x$ and $y$ both have a logarithmic scale.}
\label{fig:time:rr}
\end{figure*}
%
In this case, Perl 6 is considerably fast, although not the fastest,
and the time needed is independent of the chromosome length, a good
trait, overall. Once again, it shows a good performance in this
operation. Let us examine the next genetic operator, crossover.

The crossover performance comparison chart is shown in Figure
\ref{fig:time:xo}. In this case, after initial tests, we have gone
back again to testing a different representation: a bit string, that
is, a string composed of 0s and 1s. Strings have a different internal
representation than vectors, and the operations needed are
different. While in the first case we could use this line to perform the crossover:
\begin{lstlisting}[language=perl]
@chromosome2.splice($start,$this-len, 
    @chromosome1[$start..
      ($start+$this-len)]);
\end{lstlisting}
, in the second case we used
\begin{lstlisting}[language=perl]
$chromosome2.substr-rw($start,$this-len ) = 
    $chromosome1.substr($start,$this-len);
\end{lstlisting}
changing from an array operation to a string operation. And we did so
after finding a very disappointing performance, in fact the worst of
all languages tested, with the first one. Using a bitstring was not
much better, still needing almost double the time of the second-worst
language, which is Scala in this case. The fact that these two
functional languages have the same disappointing performance, while
Scala is usually very fast for all applications, points to the fact
that we might be taking the wrong, non-functional, approach to this
operation in these languages. 

In fact, changing the line to 
\begin{lstlisting}[language=perl]
@chromosome2.splice($start,$this-len, 
    @chromosome1.skip($start).head($this-len));
\end{lstlisting}
somewhat improved the performance. In this case we are using
functional methods to access different parts of the chromosome. There
is around a 20\% improvement over the previous line, but still very
slow compared to other languages. This proves, anyway, that testing
and some help from the community are needed to extract the best
performance out of a language; also that idiomatic constructions are
in general preferred over generic constructions. It always pays to
know the language well.

That is also why we have tested another function, the well known Royal
Road, which was proposed as an example of a complicated landscape for
evolutionary algorithms. It might also be a complicated performance
benchmark. Perl 6 needs a single line to implement this function:

\begin{lstlisting}[language=perl]
my $royal-road= $ones.rotor(4)
    .grep( so (*.all == True|False) ).elems ;	
\end{lstlisting}

In this case, we are using several unique Perl 6 features and doing so
in a functional way. For instance, {\tt | } are {\em Junctions} and
{\tt all} becomes {\tt True} or {\tt False} if all the elements in its
4-element block are. That is a very straightforward, and mathematically
correct, way to express the Royal Road function. However, it is still
slower than Perl by an order of magnitude, as shown in Figure
\ref{fig:time:rr}. In fact, we had to stop the benchmark, since
scaling with size was very bad too.

That is why we used again the {\tt .race} method, which distributes
load among threads. Although for smaller sizes the overhead needed to
set up the distribution of tasks made it slower, and thus not very
convenient for the usual sizes, it became much faster, by almost an
order of magnitude, for bigger sizes, proving again that implicit
parallelism very conveniently allows to work with big sets of
elements, making the result faster. However, it is still very slow. As
it becomes the target of optimization in subsequent releases of Perl
6, it will probably improve in speed. The implicit parallel facilities
of Perl 6 makes it possible, however, to optimize it at a different
level, for instance, population level, which still makes Perl 6 an
interesting target for the implementation of evolutionary
algorithms. In fact, there are already two implementations available
in the Perl 6 module ecosystem, one by the author of this paper, {\tt
  Algorithm::Evolutionary::Simple}, which includes implementations of
the operators shown here. The other one, {\tt Algorithm::Genetic},
makes extensive use of Perl 6 functionalities including roles and {\tt
  gather/take} loops. 

\section{Conclusions}

In this paper, we set to prove the readiness of Perl 6, a new
programming language, for implementing evolutionary
algorithms. Traditionally, these tests have been based purely on
performance, to the point that the only questions asked when a new
evolutionary algorithm library is released is: "Is it faster than
Java/C++?". In this paper we have considered this performance, first
historically from the first releases, and then considering the latest
releases. Taking into account the improvements in performance
experimented along this time, and how seriously performance issues are
taken by the developers, we can safely assume that in the medium term
Perl 6 will achieve levels of speed comparable with those of other
scripting languages, which means that it could be faster than some
compiled languages.

On the other hand, a very important consideration is also the
facilities that the language offers for the implementation of most
classical evolutionary functions. In this case, Perl 6 offers
functional methods that allow the chaining of operations, equivalent to
function composition, so that in many cases a single line
of chained functions is enough to process chromosomes. In many cases, this
idiomatic way of doing those operations will result in a faster
operation, since idiomatic constructs are usually optimized in every
language. In this sense, using either functions or the implicitly
parallel methods such as {\tt .race} results in improvements in speed,
although for the time being, and in general, Perl 6 is still slower than its
sister language, Perl.

Putting both things in the balance, and in general, the conclusion is
that the time for implementation of evolutionary algorithms in Perl 6
has arrived, although there is still some way to go in terms of
performance. Closely following the development will make the programmer
choose the faster alternative for the implementation of evolutionary
algorithms, constituting an interesting and promising line of work.

Another line of work will be to use explicit concurrency primitives to
implement a concurrent evolutionary algorithms. This is something we
will explore in a different paper. 


\begin{acks}

This paper is part of the open science effort at the university of
Granada. It has been written using {\tt knitr}, and its source as well as
the data used to create it can be downloaded from 
\href{https://github.com/JJ/2016-ea-languages-wcci}{the GitHub
  repository} \url{https://github.com/JJ/2016-ea-languages-wcci/}. 

This paper has been supported in part by
\href{http://geneura.wordpress.com}{GeNeura Team}, 
projects TIN2014-56494-C4-3-P (Spanish Ministry of Economy and
Competitiveness) and DeepBio (TIN2017-85727-C4-2-P)

We are also deeply grateful to the Perl 6 community, who through the
Perl 6 IRC channel and pull requests have helped greatly to improve
the code. 
\end{acks}

\bibliographystyle{ACM-Reference-Format}
\bibliography{geneura,languages,GA-general}

\end{document}


%%% Local Variables:
%%% ispell-local-dictionary: "english"
%%% hunspell-local-dictionary: "english"
%%% End:
